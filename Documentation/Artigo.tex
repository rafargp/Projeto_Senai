% 
% Modelo de artigo científico para a Faculdade de Tecnlogia SENAI "Mariano Ferraz"
%
% Curso de Pós-Graduação em Internet das Coisas
%
% e-mail dos integrantes do grupo:
% Rafael Gomes de Paula: rpaula@senaisp.edu.br


\documentclass[
	% -- opções da classe memoir --
	article,			% indica que é um artigo acadêmico
	12pt,				% tamanho da fonte
	oneside,			% para impressão apenas no verso. Oposto a twoside
	a4paper,			% tamanho do papel. 
	% -- opções do pacote babel --
	english,			% idioma adicional para hifenização
	brazil,				% o último idioma é o principal do documento
	sumario=tradicional
	]{abntex2}


% ---
% PACOTES
% ---

% ---
% Pacotes fundamentais 
% ---
\usepackage{lmodern}			% Usa a fonte Latin Modern
\usepackage[T1]{fontenc}		% Selecao de codigos de fonte.
\usepackage[utf8]{inputenc}		% Codificacao do documento (conversão automática dos acentos)
\usepackage{nomencl} 			% Lista de simbolos
\usepackage{color}				% Controle das cores
\usepackage{graphicx}			% Inclusão de gráficos
\usepackage{microtype} 			% Para melhorias de justificação
\usepackage{float}				% Para ajuste na posição de figuras e tabelas
% ---
		
% ---
% Pacotes adicionais, usados apenas no âmbito do Modelo Canônico do abnteX2
% ---
\usepackage{lipsum}				% para geração de dummy text
% ---
		
% ---
% Pacotes de citações
% ---
\usepackage[brazilian,hyperpageref]{backref}	 % Paginas com as citações na bibl
\usepackage[alf]{abntex2cite}	% Citações padrão ABNT
% ---

% ---
% Configurações do pacote backref
% ---
% Usado sem a opção hyperpageref de backref
\renewcommand{\backrefpagesname}{Citado na(s) página(s):~}
% Texto padrão antes do número das páginas
\renewcommand{\backref}{}
% Define os textos da citação
\renewcommand*{\backrefalt}[4]{
	\ifcase #1 %
		Nenhuma citação no texto.%
	\or
		Citado na página #2.%
	\else
		Citado #1 vezes nas páginas #2.%
	\fi}%
% ---

% ---
% Informações de dados para CAPA e FOLHA DE ROSTO
% ---
\titulo{Projeto para localização de dispositivos bluetooth indoor}
\autor{Rafael Gomes de Paula,\and Nome do segundo autor, \and Demais autores}
\local{Brasil}
\data{} 
% ---

% ---
% Alterando o aspecto da cor azul
% ---
\definecolor{blue}{RGB}{41,5,195}
% ---

% ---
% Informações do PDF
% ---
\makeatletter
\hypersetup{
     	%pagebackref=true,
		pdftitle={\@title}, 
		pdfauthor={\@author},
    	pdfsubject={Projeto Localização Indoor},
	    pdfcreator={LaTeX with abnTeX2},
		pdfkeywords={abnt}{latex}{abntex}{abntex2}{artigo científico}, 
		colorlinks=true,       		% false: boxed links; true: colored links
    	linkcolor=blue,          	% color of internal links
    	citecolor=blue,        		% color of links to bibliography
    	filecolor=magenta,      		% color of file links
		urlcolor=blue,
		bookmarksdepth=4
}
\makeatother
% --- 

% ---
% Compila o índice
% ---
\makeindex
% ---

% ---
% Altera as margens padrões
% ---
\setlrmarginsandblock{2cm}{2cm}{*}
\setulmarginsandblock{2cm}{2cm}{*}
\checkandfixthelayout
% ---

% --- 
% Espaçamentos entre linhas e parágrafos 
% --- 

% O tamanho do parágrafo é dado por:
\setlength{\parindent}{.5cm}

% Controle do espaçamento entre um parágrafo e outro:
\setlength{\parskip}{0.2cm}  % tente também \onelineskip

% Espaçamento simples
\SingleSpacing
% ---

% --- 
% Cabeçalho 
% --- 
\makepagestyle{meuestilo}
  \makeoddhead{meuestilo} %%pagina ímpar ou com oneside
     {Faculdade de Tecnologia SENAI "Mariano Ferraz"}
     %{Vol. 1,  n\textsuperscript{o} 1 (2017)}
     {}
     {IoT - SSIR,  pág. \thepage}
% ---

% ---
% Margem para resumo, palavras-chave, abstract e keywords
% ---
\def\changemargin#1#2{\list{}{\rightmargin#2\leftmargin#1}\item[]}
\let\endchangemargin=\endlist 
% ----

% ---
% Início do documento
% ---
\begin{document}

% ----------------------------------------------------------
% ELEMENTOS TEXTUAIS
% ----------------------------------------------------------
\textual

% Aplica o cabeçalho em todas as páginas, excetuando-se a primeira
\pagestyle{meuestilo}

% Retira espaço extra obsoleto entre as frases.
\frenchspacing 

% Página de titulo
\maketitle

% Aplica cabeçalho na primeira página
\thispagestyle{meuestilo}
% ---

% -----------------------------------------------------------
% Resumo em português
% -----------------------------------------------------------
\begin{changemargin}{1cm}{1cm} 
 \textbf{Resumo} – A proposta do projeto é criar um sistema de localizaçao em tempo real de dispositivos bluetooth dentro de um local fechado.
 O hardware consiste na criação de um dispositivo utilizando o ESP32, capaz de localizar outros dispositivos bluetooth, como IBeacons, através da triangulação destes dispositivos que denominamos "estações de rastreamento". 
 As estações disponibilizaram as informações dos dispositivos encontrados através da internet, utilizando o protocolo MQTT e a plataforma Node-Red.
 O Software consiste em um website, construído utilizando Node.js como servidor aplicacional, capaz de se conectar ao Node-Red e fazer a interpretação dos dados.
 A plataforma é responsável por receber os dados e determinar qual é o dispositivo mais proximo dentro do raio da triangulação das estações.
 \vspace{\onelineskip}
 
 \noindent
 \textbf{Palavras-chave} – artigo científico; formatação; Internet das Coisas 
\end{changemargin}

% ---

% ----------------------------------------------------------
% Introdução
% ----------------------------------------------------------
\section{Introdução}
\addcontentsline{toc}{section}{Introdução}

% ---
% Nota de rodapé na primeira página
% ---
%\let\thefootnote\relax\footnotetext{(A ser preenchido pelos editores)  Versão inicial submetida em DD de MM. de AAAA.   Versão final aceita em DD de MM. de AAAA.  Publicado em DD de MM. de AAA.  Digital Object Identifier \_\_\_\_\_\_\slash\_\_\_\_\_ }
\let\thefootnote\svthefootnote
% ---

  Esse artigo foi preparado com TeXstudio versão 2.11 e Overleaf. Deixe o primeiro parágrafo de uma seção, sub-seção ou item sem tabulação. Os demais parágrafos podem ser tabulados.  Deixe o texto dos parágrafos justificados, isto é, alinhados à esquerda e a direita, como já é feito automaticamente pelo LaTeX.
    
	Use apenas a fonte proporcional com serifas “Computer Modern Unicode Serif (CMU Serif)”, padrão do LaTeX, para escrever o texto do artigo.  O tamanho de fonte padrão é 12 pt.  Use esse tamanho em todo o corpo do artigo, a menos que outro tamanho seja indicado (como é o caso dos títulos de seções e sub-seções).
    
	O título do artigo deve aparecer 24 pt (ou duas linhas) abaixo do topo da primeira página, centrado e usando fonte tamanho 14 pt (sem negrito ou sublinhado).  Coloque a primeira letra de cada palavra em maiúscula (com exceção de partículas como artigos e preposições).
    
	Deixe um espaço de 12 pt (ou uma linha) em branco após o título e inclua a lista de nomes de autores(as) separados por vírgulas, centrada na página.  Se necessário, use mais de uma linha para listar todos os nomes.  Não inclua dados de filiação ou títulos na primeira página: há um espaço reservado para mini-currículos ao final do artigo.
    
	O resumo do artigo e a lista de palavras-chave vêm em seguida, separadas por espaços de 12 pt, como mostra o exemplo acima.  Esta última começa com “Palavras-chave”, em negrito, seguida de um traço e da lista separada por ponto-e-vírgula.

\section{Seção}

Deixe um espaço em branco de 12 pt (ou uma linha) antes e depois do título de uma seção.  As seções devem ser numeradas sequencialmente com algarismos arábicos. 

O título da seção deve usar fonte tamanho 14 pt (sem negrito nem itálico).

\subsection{Subseção}

Seções podem ser divididas em subseções.  Deixe um espaço de 12 pt antes do título da subseção e 6 pt (ou meia linha) depois.  Numere as subseções com o número da seção e da subseção, separados por um ponto, iniciando as subseções por ‘1’ em cada seção.  O título da subseção deve usar fonte tamanho 12 pt (sem negrito nem itálico).

Esta é a estrutura preferencial do artigo, com apenas 2 níveis: seção e subseção.  Procure organizar seu artigo desta forma.

\begin{enumerate}[label=\Alph*]
\item \textit{Subitem}

Se realmente for necessário usar um nível a mais, divida a subseção em subitens.  Deixe um espaço de 12 pt antes do título do subitem e nenhum espaço depois.  Numere os subitens com letras maiúsculas, iniciando por A em cada subseção.   O título do subitem deve usar fonte tamanho 12 pt, em itálico (sem negrito).

Por favor, não estruture seu artigo com mais do que três níveis de hierarquia (ou seja, não crie um sub-subitem).  Se um do seu artigo subitem está tão longo a ponto de ser necessário dividi-lo, é sinal que ele deve constituir uma subseção ou até mesmo uma nova seção.

\end{enumerate}

\section{Figura}

Figuras devem ser centradas na página e numeradas com algarismos arábicos.  A legenda segue abaixo e numerada. Separe a figura do texto anterior com um espaço de 12 pt. 

A figura deve estar na mesma pasta do arquivo, ou ser adicionada no Overleaf, clicando em project e adicionando a imagem.

% \begin{figure}[H]     % Com o pacote "float", o "[H]" fixa a figura nesse lugar 
%     \centering
%     \includegraphics[width=0.5\textwidth]{figure}
%     \caption{figura}
%     \label{figura}
% \end{figure}

Após a legenda, deixe um espaço em branco de 12 pt para o texto seguinte.

\section{Tabelas}

Tabelas devem ser centradas na página e numeradas com algarismos arábicos.  A legenda deve ficar acima da tabela.   Linhas verticais são opcionais.  Separe a figura do texto anterior com um espaço de 12 pt. 

\begin{table}[H]     % Com o pacote "float", o "[H]" fixa a figura nesse lugar
\centering
\caption{Legenda da tabela}
\label{my-label}

\begin{tabular}{c|c|c|c}
Data       & Versão & Descrição   & Autor  \\ \hline
12/09/2015 & 1      & Descrição 1 & Nome 1 \\
05/12/2015 & 2      & Descrição 2 & Nome 2
\end{tabular}
\end{table}

Após a legenda, deixe um espaço em branco de 12 pt para o texto seguinte. 

\section{Equações}

Equações devem separadas do texto por espaços de 6 pt (meia linha) antes e depois.  Numere as equações com algarismos arábicos, colocados entre parênteses e alinhados à esquerda,  como mostra o exemplo a seguir.

\begin{equation}
x = y+1.
\end{equation}

Pontue a equação e continue o texto seguinte como se a equação fosse parte do texto.  Por exemplo, a equação 1 termina com um ponto final e o parágrafo seguinte a ela (que é este) começa em maiúscula e com tabulação.  Segundo exemplo: em

\begin{displaymath}
y = x-1,
\end{displaymath}

segue-se a equação 2 com uma vírgula, e o parágrafo seguinte (que é este) começa em minúscula e sem tabulação. Utilize "\$"  para inserir equações no meio do texto, como em $y=x-2$.

\section{Referências Bibliográficas}

Referências devem ser citadas no texto no formato (SOBRENOME, ano) ou (SOBRENOME1 e SOBRENOME2, ano), entre parênteses, com os sobrenomes dos autores em maiúscula e o ano de publicação com 4 dígitos.  Havendo muitos autores, indique o primeiro e abrevie os demais com “et al.” – por exemplo (FULANO et al., 1998).

Caso deseje citar uma obra forma textual no artigo, inclua o ano da publicação entre parênteses após o(s) nome(s) dos autor(es).  Por exemplo: “Segundo Fulano e Sicrano (1999), é possível que...”.

%Para mais informações sobre como citar adequadamente diferentes tipos de trabalho, consulte por exemplo as diretrizes compiladas pela Divisão de Bibliotecas da Escola Politécnica da USP (2013), disponível em

%http://www.poli.usp.br/images/stories/media/download/bibliotecas/DiretrizesTesesDissertacoes.pdf

\appendix
\section*{Apêndices}
 
A seção de apêndices é opcional.  Se precisar incluí-la, deixe 24 pt (duas linhas) de espaço antes do título “Apêndices”,  escrito centrado na linha, em letras maiúsculas (ou versalete) e tamanho 14 pt.
 
\addcontentsline{toc}{section}{Apêndices}   % Definindo seção de apêndice
\renewcommand{\thesection}{\textit{\Alph{section}} }
\section{\textit{Primeiro apêndice}}
 
Deixe um espaço de 12 pt antes e depois do título do apêndice.  Escreva o título em tamanho 14 pt e em itálico.  Os apêndices devem ser numerados com letras maiúsculas.


\section*{Agradecimentos}

Agradecimentos (opcional).  Deixe um espaço de 24 pt antes do título “Agradecimentos”, escrito centrado na linha, em maiúscula (ou versalete) e tamanho 14 pt.  Escreva o texto dos agradecimentos após um espaço de 12 pt.

\section*{Referências}

O título da seção de referências deve ter uma espaço antes de 24 pt e deve ser escrito centrado na linha, em maiúscula (ou versalete) e tamanho 14 pt.  Deixe um espaço de 12 pt antes do texto da primeira referência.  Deixe um espaço de 6 pt (meia linha) entre as referências.
Liste as obras pela ordem alfabética do sobrenome do primeiro autor (ou nome da empresa).  Não numere as referências.  Alguns exemplos:
SOBRENOME, Nome. Título da publicação.  Edição, Cidade de publicação: Editora, ano, páginas $\{ $Referencia$\} $.

SOBRENOME1, Nome1;  SOBRENOME2, Nome2.  Título da publicação.  Edição, Cidade de publicação: Editora, ano, páginas.
SOBRENOME1, Nome1;  SOBRENOME2, Nome2; et al.  Título da publicação.  Edição, Cidade de publicação: Editora, ano, páginas.

(Se seu artigo não for escrito em inglês, acrescente aqui as versões em inglês do título, resumo e palavras-chaves)

\begin{changemargin}{1cm}{1cm} 

Title: Template for Internet of Things

\textbf{Abstract} – This is the English version of the resume.

\textbf{Keywords} – Comma separated list of keywords.

\end{changemargin}

% \begin{figure}[h]
%     \includegraphics[width=0.5\textwidth]{figure2}
%     \caption{figura2}
%     \label{figura2}
% \end{figure}

\textbf{Nome do Autor 1}, em negrito, seguido de seu minicurrículo. Foto 3 cm x 4 cm opcional, emoldurada pelo texto do minicurrículo.

% \begin{figure}[h]
%     \includegraphics[width=0.5\textwidth]{figure2}
%     \caption{figura2}
%     \label{figura2}
% \end{figure}

\textbf{Nome do Autor 2}, seguido de seu minicurrículo. Foto 3 cm x 4 cm opcional, emoldurada pelo texto do minicurrículo.


\end{document}